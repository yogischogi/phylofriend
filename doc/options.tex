\section{Command Line Options}

Command line options may be given in arbitrary order.

\begin{description}
\item[-help] Prints available program options.
\item[-personsin] Filename or directory of files containing the
	persons' Y-STR values. If this is a single file it must contain
	results for multiple persons. The input file format is CSV
    (comma separated values) or text format.

	If a directory is provided for input it must contain multiple
	files in YFull format, each file containing the results for
	a single person. The person's ID is extracted from the filename.

	\emph{personsin} supports multiple file names separated by
	commas.
\item[-labelcol] Number of the column that is used for labels
	when reading CSV files.
\item[-mrin] Filename of the mutation rates to use.
\item[-anonymize] If this is true persons' names are replaced by numbers.
\item[-modal] Creates modal haplotype and performs TMRCA calculation.
\item[-phylipout] Filename for the distance matrix that can be fed into
	the PHYLIP\cite{Phylip} program.
\item[-mrout] Filename for the output of the currently used mutation rates.
\item[-txtout] Filename for text output of persons and Y-STR values.
\item[-htmlout] Filename for HTML output of persons and Y-STR values.
\item[-nmarkers] Uses only the given number of markers for calculations.
\item[-gentime] Generation time.
\item[-cal] Calibration factor.
\item[-reduce] Reduces the number of persons by the given factor
	 (for large numbers of samples).
\item[-statistics] Prints marker statistics.
\end{description}

