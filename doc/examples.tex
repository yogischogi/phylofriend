\section{Examples}

\subsection{Create a Phylogenetic Tree}

\begin{enumerate}
\item Copy persons' data from a Family Tree DNA project website into
	a spreadsheet. If the Y-STR values do not appear properly try
	inserting them into the spreadsheet as unformatted text.
\item Save the spreadsheet in CSV (comma separated values)
	format, for example \emph{persons.csv}.
\item Start a terminal or command line interpreter and go
	to the directory where you stored \emph{persons.csv}.
\item Create a matrix of genetic distances by typing\\
	\texttt{phylofriend -personsin persons.csv -phylipout infile\\
	-mrin 67-average.txt}\\
	Here we use a set of average mutation rates for 67 markers.
    The file
	\emph{67-average.txt} can be found in the directory
	\emph{phylofriend/mutationrates/}.
    The result is a matrix that contains the number of generations
    as a measure for genetic distances between the persons.
\item Use the PHYLIP program to create a tree in Newick format
	with\\
	\texttt{/usr/lib/phylip/bin/kitsch}\\
	You will need to answer some questions. Usually the
	default values are good enough. The results will be
	two text files, one named \emph{outtree} which contains
	the tree in Newick format and another one named
	\emph{outfile} which contains a more human readable
	description.
\item Create an image of the tree by typing\\
	\texttt{/usr/lib/phylip/bin/drawgram}\\
	Use \emph{outtree} as the input file name.
	The resulting tree will be stored in a file named 
	\emph{plotfile}.

	A nice alternative to visualize the tree is the use of the
	\href{http://www.trex.uqam.ca/index.php?action=newick&project=trex}
	{Trex}\cite{Trex}
	webserver. You can copy the contents of the file \emph{outtree}
	into the Trex window.
\end{enumerate}


\subsection{Pimp Your Tree with Nicer Labels}

By default Phylofriend assumes that your persons input file's
first column contains a list of IDs. This is usually a Family
Tree DNA Kit number. The resulting tree is hard to read. Many
projects keep names in another column. You can access this 
column by using the \emph{labelcol} option. Suppose your second
column contains names. You can create a distance matrix with
names instead of IDs by typing

\noindent\texttt{phylofriend -personsin persons.csv -labelcol 2\\
-phylipout infile -mrin 67-average.txt}

Due to compatibility issues with other programs the labels
must be 10 characters long and may only contain 8-bit
characters. Phylofriend will apply a transformation to
make sure that the requirements are fulfilled 
but the result is sometimes a bit strange.

You can also use the \emph{labelcol} option to create trees
that contain the origins of people or the haplogroups. Although
I strongly recommend to build trees only from people who
belong to the same haplogroup this is sometimes useful if
you want to know if different haplogroups are close on their
Y-STR values.

If you want to publish your tree you will often need to
protect the privacy of the members. This is what the
\emph{anonymize} option is for. By typing

\noindent\texttt{phylofriend -personsin persons.csv -phylipout infile -anonymize\\
-mrin 67-average.txt}

you will get a distance matrix where the names are replaced
by numbers.


\subsection{Use a Specific Set of Mutation Rates}

Phylofriend supports the use of arbitrary mutation rates by
the \emph{mrfile} option. The \emph{phylofriend/mutationrates}
directory contains some files with mutation rates. The average
mutation rates where taken from \cite{Kly12}.
If you like to compare on 67 markers or 111 markers you can use

\noindent\texttt{phylofriend -personsin persons.csv -phylipout infile\\
-mrin 67-average.txt}

\noindent or

\noindent\texttt{phylofriend -personsin persons.csv -phylipout infile\\
-mrin 111-average.txt}


\subsection{Calibrate Your Data}

Mutation rates depend on the method applied to calculate
genetic distances and the sample populations used. Mutations
themselves occur by coincidence. Average mutation rates
often yield acceptable results but in most cases you will
have to calibrate your data, especially if you want to
calculate genetic distances in years.

Phylofriend provides two options for data calibration:
\emph{gentime}, the generation time in years and
\emph{cal} an additional calibration factor. Internally
they are just multiplied together but using two separate
factors seems more convenient for typical use cases.

A generation time of 32 years has proven to show good
results \cite{YFullMutationRate}. You can use it by typing

\noindent\texttt{phylofriend -personsin persons.csv -phylipout infile\\
-gentime 32 -mrin 67-average.txt}

In reality an optimal result is often hard to achieve.
Especially within the range of genealogical time frames
(about 400 years) and only small numbers of persons who
have tested, you are often left with a large statistical error.
Even worse, the method of Y-STR counting has numerous
pitfalls, some of which are discussed in \cite{Ham15}.

If you have a reliable paper trail or a well defined
historic event you can calibrate your data using the
\emph{cal} option. With \emph{cal} you just provide an
additional calibration factor that is multiplied to
the calculated genetic distances, for example

\noindent\texttt{phylofriend -personsin persons.csv -phylipout infile\\
-gentime 32 -mrin 67-average.txt -cal 1.2}

multiplies all genetic distances by a factor of 1.2.



\subsection{Count Mutations}

If you want to count mutational differences between
persons, for example on a 37 marker scale, you can
achieve this by typing

\noindent\texttt{phylofriend -personsin persons.csv -phylipout distancecount.txt\\
-nmarkers 37 -cal 37}

The \emph{nmarkers} option restricts all calculations
to the given set of markers (usually 37, 67 or 111). If
a person has not tested for enough markers, he is excluded
from the calculation. Because Phylofriend uses average
values internally, all results are devided by 37 in 
the previous example. To get whole numbers we must compensate
for this effect. This is done by multiplying the results
with 37 using the \emph{cal} option.

The internal use of average values seems complicated at
first but it allows Phylofriend to compare persons who
have tested for different numbers of markers.

Phylofriend's results may also differ from other programs
because Phylo\-friend uses it's own hybrid mutation model.

An alternative approach to mutation counting is using a set
of mutation rates that was especially created for that purpose,
for example:

\noindent\texttt{phylofriend -personsin persons.csv -phylipout distancecount.txt\\
-mrin 37-count.txt}

The difference to the previous example is that the \emph{nmarkers}
option in the previous example makes sure that all persons
have tested for all 37 markers, while the use of the
\emph{37-count} mutation rate also accepts persons who have
tested for a lower number of markers. The result is an estimate.
This option was introduced because next generation sequencing
has revealed Y-STR results for 587 markers. However due to
the technical restrictions of this method, different persons
usually get results for different, but largely overlapping,
marker sets.


\subsection{Marker Statistics}

Phylofriend can give you detailed statistics about marker
values. It calculates the minimal and maximal value for each
marker and the number of occurrences. You can activate the
statistical output by using the \texttt{-statistics=true}
option, for example

\vspace{1ex}
\noindent\texttt{phylofriend -personsin=s11481.csv -statistics=true -nmarkers=6}
\vspace{1ex}

\noindent prints out the results for the first six markers
(\texttt{-nmarkers=6}). In our example it is assumed that
the persons' data is in a file called \emph{s11481.csv}.
This is the output:

\begin{verbatim}
DYS393, Min: 13, Max: 13, 13:15,
DYS390, Min: 24, Max: 24, 24:15,
DYS19, Min: 14, Max: 14, 14:15,
DYS391, Min: 10, Max: 11, 11:14, 10:1,
DYS385a, Min: 11, Max: 12, 11:10, 12:5,
DYS385b, Min: 12, Max: 14, 12:1, 14:14,
\end{verbatim}

\noindent
Each line contains the marker name and the minimal and
maximal values. After that each marker value is listed
together with it's frequency, separated by a colon.

Let's take DYS391 as an example. It has two values,
10 and 11. The value 10 occurs 1 time and the value
11 occurs 14 times.


\subsection{Use Data from YFull}

\href{http://yfull.com/}{YFull} reports up to 587 STR values
(March 2018), but not all of them for each person. A Family
Tree DNA Big Y test usually reveals about 500 STR markers,
some times less, some times more.

To use these results put the YFull result files for every
single person into a directory, for example \emph{inputdir}.
Call Phylofriend by typing:

\vspace{1ex}
\noindent\texttt{phylofriend -personsin inputdir -phylipout infile -modal\\
-mrin 587-count.txt}
\vspace{1ex}

\noindent
This command will use up to 587 markers for comparison.
The effect of the file \emph{587-count.txt} is that the 
mutation differences are just counted. An average mutation
rate for 587 markers is not available.


\subsection{Extract Data from a Spreadsheet in CSV format}

Spreadsheet data is often uncomfortable to handle, especially
if you want to write your own program and need to parse it.
For this purpose Phylofriend supplies the \emph{txtout}
option. It writes data to a text file in simplified form.

The easiest way to use it is

\vspace{1ex}
\noindent\texttt{phylofriend -personsin persons.csv -txtout persons.txt}
\vspace{1ex}

\noindent
This extracts the data from \emph{persons.csv} and writes
it to \emph{persons.txt}. The first column of \emph{persons.txt}
contains the first column found in \emph{persons.csv}, usually
a set of IDs. The following columns contain the Y-STR values.
All columns are separated by tabs.

If you want to use another column you can use the
\emph{labelcol} option:

\vspace{1ex}
\noindent\texttt{phylofriend -personsin persons.csv -txtout persons.txt\\
-labelcol 2}
\vspace{1ex}

\noindent
This extracts the second column from \emph{persons.csv} and
writes it to the first column of \emph{persons.txt}.

When using the \emph{txtout} option you will need to specify
how many Y-STR values are written to the text file. This is
done by \emph{nmarkers}. Phylofriend will write the same number
of Y-STR values to each line. Missing values are written as
0, larger sets of Y-STR values are truncated. This is how to
write a full set of 111 markers:

\vspace{1ex}
\noindent\texttt{phylofriend -personsin persons.csv -txtout persons.txt\\
-nmarkers 111}
















